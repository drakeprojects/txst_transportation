% Options for packages loaded elsewhere
% Options for packages loaded elsewhere
\PassOptionsToPackage{unicode}{hyperref}
\PassOptionsToPackage{hyphens}{url}
\PassOptionsToPackage{dvipsnames,svgnames,x11names}{xcolor}
%
\documentclass[
  letterpaper,
  DIV=11,
  numbers=noendperiod]{scrartcl}
\usepackage{xcolor}
\usepackage{amsmath,amssymb}
\setcounter{secnumdepth}{5}
\usepackage{iftex}
\ifPDFTeX
  \usepackage[T1]{fontenc}
  \usepackage[utf8]{inputenc}
  \usepackage{textcomp} % provide euro and other symbols
\else % if luatex or xetex
  \usepackage{unicode-math} % this also loads fontspec
  \defaultfontfeatures{Scale=MatchLowercase}
  \defaultfontfeatures[\rmfamily]{Ligatures=TeX,Scale=1}
\fi
\usepackage{lmodern}
\ifPDFTeX\else
  % xetex/luatex font selection
\fi
% Use upquote if available, for straight quotes in verbatim environments
\IfFileExists{upquote.sty}{\usepackage{upquote}}{}
\IfFileExists{microtype.sty}{% use microtype if available
  \usepackage[]{microtype}
  \UseMicrotypeSet[protrusion]{basicmath} % disable protrusion for tt fonts
}{}
\makeatletter
\@ifundefined{KOMAClassName}{% if non-KOMA class
  \IfFileExists{parskip.sty}{%
    \usepackage{parskip}
  }{% else
    \setlength{\parindent}{0pt}
    \setlength{\parskip}{6pt plus 2pt minus 1pt}}
}{% if KOMA class
  \KOMAoptions{parskip=half}}
\makeatother
% Make \paragraph and \subparagraph free-standing
\makeatletter
\ifx\paragraph\undefined\else
  \let\oldparagraph\paragraph
  \renewcommand{\paragraph}{
    \@ifstar
      \xxxParagraphStar
      \xxxParagraphNoStar
  }
  \newcommand{\xxxParagraphStar}[1]{\oldparagraph*{#1}\mbox{}}
  \newcommand{\xxxParagraphNoStar}[1]{\oldparagraph{#1}\mbox{}}
\fi
\ifx\subparagraph\undefined\else
  \let\oldsubparagraph\subparagraph
  \renewcommand{\subparagraph}{
    \@ifstar
      \xxxSubParagraphStar
      \xxxSubParagraphNoStar
  }
  \newcommand{\xxxSubParagraphStar}[1]{\oldsubparagraph*{#1}\mbox{}}
  \newcommand{\xxxSubParagraphNoStar}[1]{\oldsubparagraph{#1}\mbox{}}
\fi
\makeatother


\usepackage{longtable,booktabs,array}
\usepackage{calc} % for calculating minipage widths
% Correct order of tables after \paragraph or \subparagraph
\usepackage{etoolbox}
\makeatletter
\patchcmd\longtable{\par}{\if@noskipsec\mbox{}\fi\par}{}{}
\makeatother
% Allow footnotes in longtable head/foot
\IfFileExists{footnotehyper.sty}{\usepackage{footnotehyper}}{\usepackage{footnote}}
\makesavenoteenv{longtable}
\usepackage{graphicx}
\makeatletter
\newsavebox\pandoc@box
\newcommand*\pandocbounded[1]{% scales image to fit in text height/width
  \sbox\pandoc@box{#1}%
  \Gscale@div\@tempa{\textheight}{\dimexpr\ht\pandoc@box+\dp\pandoc@box\relax}%
  \Gscale@div\@tempb{\linewidth}{\wd\pandoc@box}%
  \ifdim\@tempb\p@<\@tempa\p@\let\@tempa\@tempb\fi% select the smaller of both
  \ifdim\@tempa\p@<\p@\scalebox{\@tempa}{\usebox\pandoc@box}%
  \else\usebox{\pandoc@box}%
  \fi%
}
% Set default figure placement to htbp
\def\fps@figure{htbp}
\makeatother


% definitions for citeproc citations
\NewDocumentCommand\citeproctext{}{}
\NewDocumentCommand\citeproc{mm}{%
  \begingroup\def\citeproctext{#2}\cite{#1}\endgroup}
\makeatletter
 % allow citations to break across lines
 \let\@cite@ofmt\@firstofone
 % avoid brackets around text for \cite:
 \def\@biblabel#1{}
 \def\@cite#1#2{{#1\if@tempswa , #2\fi}}
\makeatother
\newlength{\cslhangindent}
\setlength{\cslhangindent}{1.5em}
\newlength{\csllabelwidth}
\setlength{\csllabelwidth}{3em}
\newenvironment{CSLReferences}[2] % #1 hanging-indent, #2 entry-spacing
 {\begin{list}{}{%
  \setlength{\itemindent}{0pt}
  \setlength{\leftmargin}{0pt}
  \setlength{\parsep}{0pt}
  % turn on hanging indent if param 1 is 1
  \ifodd #1
   \setlength{\leftmargin}{\cslhangindent}
   \setlength{\itemindent}{-1\cslhangindent}
  \fi
  % set entry spacing
  \setlength{\itemsep}{#2\baselineskip}}}
 {\end{list}}
\usepackage{calc}
\newcommand{\CSLBlock}[1]{\hfill\break\parbox[t]{\linewidth}{\strut\ignorespaces#1\strut}}
\newcommand{\CSLLeftMargin}[1]{\parbox[t]{\csllabelwidth}{\strut#1\strut}}
\newcommand{\CSLRightInline}[1]{\parbox[t]{\linewidth - \csllabelwidth}{\strut#1\strut}}
\newcommand{\CSLIndent}[1]{\hspace{\cslhangindent}#1}



\setlength{\emergencystretch}{3em} % prevent overfull lines

\providecommand{\tightlist}{%
  \setlength{\itemsep}{0pt}\setlength{\parskip}{0pt}}



 


\KOMAoption{captions}{tableheading}
\makeatletter
\@ifpackageloaded{caption}{}{\usepackage{caption}}
\AtBeginDocument{%
\ifdefined\contentsname
  \renewcommand*\contentsname{Table of contents}
\else
  \newcommand\contentsname{Table of contents}
\fi
\ifdefined\listfigurename
  \renewcommand*\listfigurename{List of Figures}
\else
  \newcommand\listfigurename{List of Figures}
\fi
\ifdefined\listtablename
  \renewcommand*\listtablename{List of Tables}
\else
  \newcommand\listtablename{List of Tables}
\fi
\ifdefined\figurename
  \renewcommand*\figurename{Figure}
\else
  \newcommand\figurename{Figure}
\fi
\ifdefined\tablename
  \renewcommand*\tablename{Table}
\else
  \newcommand\tablename{Table}
\fi
}
\@ifpackageloaded{float}{}{\usepackage{float}}
\floatstyle{ruled}
\@ifundefined{c@chapter}{\newfloat{codelisting}{h}{lop}}{\newfloat{codelisting}{h}{lop}[chapter]}
\floatname{codelisting}{Listing}
\newcommand*\listoflistings{\listof{codelisting}{List of Listings}}
\makeatother
\makeatletter
\makeatother
\makeatletter
\@ifpackageloaded{caption}{}{\usepackage{caption}}
\@ifpackageloaded{subcaption}{}{\usepackage{subcaption}}
\makeatother
\usepackage{bookmark}
\IfFileExists{xurl.sty}{\usepackage{xurl}}{} % add URL line breaks if available
\urlstyle{same}
\hypersetup{
  pdftitle={Transportation\_survey\_journal},
  pdfauthor={drakeprojects},
  colorlinks=true,
  linkcolor={blue},
  filecolor={Maroon},
  citecolor={Blue},
  urlcolor={Blue},
  pdfcreator={LaTeX via pandoc}}


\title{Transportation\_survey\_journal}
\author{drakeprojects}
\date{}
\begin{document}
\maketitle

\renewcommand*\contentsname{Table of contents}
{
\hypersetup{linkcolor=}
\setcounter{tocdepth}{3}
\tableofcontents
}

\section{Introduction \& Background}\label{introduction-background}

\subsection{Universities}\label{universities}

\subsubsection{Creating Demand}\label{creating-demand}

Universities campuses in college towns provide unique opportunities to
reduce transportation-related greenhouse gas emissions(GHGe) other
pollutants. Interest in university student travel behavior has grown
among researchers and policy makers and car dependency continues to
adversely impact U.S. communities{[}Zhou, Wang, and Wu
(2018){]}(\textbf{kattak2011?}). Car dependency continues to have
adverse environmental, social and economic consequences on communities.
including pollution, health effects, and financial burden(Romanowska,
Okraszewska, and Jamroz 2019). Universities are recognized institutions
that create awareness of climate change mitigation and sustainable
transportation practices{[}(\textbf{romanowsaka2019?}){]}(Angelis,
Mantecchini, and Pietrantoni 2021)(Zhou 2012).

Referred to as ``cities inside cities'' . universities are ideal
learning laboratories for researching and identifying sustainability
initiatives. Operating as a communal network that reflects a large
community and function universities have a significant role in promoting
sustainable development and tackling transportation concerns. The
movement of students, students staff and visitors is a critical factor
in the growing impact of universities campuses to the surrounding area.
Universities have a large footprint, accounting for up to 2\% of the
total annual U.S. GHGe, making it important that they participate in
mitigating climate change(\textbf{parsley2022?}).

University students are considered a key segment of the population those
travel is often understudies and weakly covered in current literature
and data collection{[}(\textbf{kattak2011?}){]}(Taylor and Mitra 2021).
Current research explores daily commuting patterns, factors influencing
travel behavior, and the potential for shifting toward more sustainable
travel modes{[}Romanowska, Okraszewska, and Jamroz
(2019){]}(\textbf{zhou2011?}).Factors affecting mode choice include
physical environment(distance, time, density), mode-specific cost,
accessibility to transit, personal attributes, trip characteristics,
travel demand management (TDM) measures, and psychological factors( or
awareness and previewed safety(Zhou, Wang, and Wu 2018).

\subsubsection{Trip Generation}\label{trip-generation}

Universities are major contributors to commuting trips averaging 20\%
higher than the general public, \textbf{3.69} daily tripsKhattak et al.
(2011). Universities serve a large number of students and staff forming
a significant portion of the local or regional population(Romanowska,
Okraszewska, and Jamroz 2019). This large concentrationed populations
moving from and around a campact campus inherently generate a
significant volumne of daily trips(Romanowska, Okraszewska, and Jamroz
2019). (Khattak et al. 2011) defines a trip as moving at least 300 feet
from one address to another. This definition is important for capturing
travel behavior of university students living on and around campus.
Undergraduate students living on campus make the most trips when
compared to other university groups. University students make shorter
non-motorized trips, utilizing walking and biking infrastructure.
(Khattak et al. 2011).

\subsubsection{Using Alternative Travel
Modes}\label{using-alternative-travel-modes}

University students transient living and wide use to travel modes,
marking them as notably from the general population. University students
are more likely to adopt alternative travel modes, having a higher
portion versus staff and general public. Students populations tend to
walk and and bike more reducing car trips. Student tend prioritize
housing that is affordable and close to campus, providing them more
travel mode choice. The variety of quality transportation modes and high
portion of alternative transportation modes creates a need for a more
comprehensive transportation system, less reliant on fossil fuels.
Universities increase the demand on local transportation system while
providing an opportunity to expand alternative transportation modes.
Today large university campuses are a microcosm of the urban landscape
and provides a ideal environment for studying transportation policy and
infrastructure. (Zhou, Wang, and Wu 2018, 133)

College towns are recognized as areas where university students account
for a significant share of the local population(33\% or more) and tend
to have smaller population sizes and a more compact urban form compared
to large urban areas(Zhou, Wang, and Wu 2018). This compactness suggest
students are more likely to walk or bike and drive less(Zhou, Wang, and
Wu 2018). A study focused on college students commuter students were
more likely to use alternative and active travel modes when compared to
staff and peers at urban universities(Zhou, Wang, and Wu 2018). However,
25\% of commuters students living within two miles of campus drove
alone, indicating that a compact form is only the only factor of
satisfying all travel needs, like out of town travel(Zhou, Wang, and Wu
2018). The advantage of shorter distances, good transit, and a safe
environment are not guarantees of alternative mode, suggesting the need
for further strategies to manage transportation needs(Zhou, Wang, and Wu
2018).

\subsubsection{Travel Factors}\label{travel-factors}

(\textbf{romanowka2019?}) identifies access to a vehicle, trip origin,
and trip distance as key factors in travel mode choice.(Angelis,
Mantecchini, and Pietrantoni 2021) finds locations and travel continuity
as significantly impact mode choice. The high density at Texas State
increases potential for ``transport infrastructure and services''
supporting multi-modal travel. Compact campus design promoted higher
trip rates among students. Higher density creates more complex
transportation patterns. May increase the importance of efficient
multi-modal transportation systems. High students density presents both
challenges and opportunities for promoting alternative transportation.
Studies find that improved biking and pedestrian infrastructure reduces
car commuting, to up 14\%(\textbf{romanowka2019?}). University students
live relatively close to campus and have accessible to quality
multi-modal transportation systems(Zhou, Wang, and Wu 2018, 132) For
many the university campus is the A more rigorous focus on university
travel will provide insight on how the built environment combined with
social factors effect travel behavior(Angelis, Mantecchini, and
Pietrantoni 2021). Universities are active;ly encouraging sustainable
education and practices(Parsley and Waliczek, n.d.). Students are a
social group with high willingness to adopt new ideas and make lifestyle
changes. The knowledge, experiences, and attitudes gain during college
raise awareness in for the wider public(Romanowska, Okraszewska, and
Jamroz 2019). Universities can leverage this to promote
pro-environmental values and prevent the development of pro-car habits
in young commuters(Angelis, Mantecchini, and Pietrantoni 2021).

\subsubsection{Window of Opportunity}\label{window-of-opportunity}

The university is a ``window of opportunity'' for significant lifestyle
changes, like travel behavior. Travel behavior habits are repetitive
decision made in a stable state of life based on the current knowledge
and perspective of available resources. Theory of planned behavior(TPB)
suggest travel behavior are habits that are significant influenced by a
commuters knowledge and perceptions of the available transportation
system. Travel behavior is understood to give insight to commuter belief
and perspectives, and attitude towards travel modes.
(\textbf{deangelis2012?}) used the Value-belief-norms(VBN) understand
that an individuals moral obligation to the environment can
significantly impact travel mode choice. Understanding that habitual
travel behaviors are based on individual morals, explains how personal
experiences and attitudes can shape travel mode choice. A commuter's
knowledge and perception of the available transportation modes will
decide their travel behavior.

Study suggest habits are best broken and formed most easily during
transitional period of life like have a child, relocating for a new job,
or going to college. During these transitional periods individuals are
inevitably induced to new environments or experiences that may influence
their perspectives and grow their knowledge base. Changes adopted during
this ``window of opportunity'' are likely to effect lifestyle choice
like mode choice and last into later life beyond the university campus.
(\textbf{deangelis2012?}). An commuters who uses alternative
transportation option during college will view biking, walking, or
public transit as favorable options later in life. Where as someone who
drive most of their time on campus is likely to continue with less
chance of reconsidering other options. (\textbf{okrazeska?},) find that
students are more likely than staff to choice alternative and active
travel modes. About 20\% University staff preferred travel modes for
their directness, duration, convenience and flexibility. These travel
preferences are consistent with staff demographics, representing a more
stable consistent lifestyle, having dependents, and living away from
campus.

\subsubsection{Research Question}\label{research-question}

How can universities leverage physical factors of campus to reduce
transportation emissions?

\subsubsection{Methodology}\label{methodology}

This study site is Texas State University(TXST), a premier public school
institution located in San Marcos Texas. As one of seven institutions in
The Texas State University System, San Marcos(main) campus serves 40,678
students today. This campus is a thriving public research institution
with an annual exceeding \$165 million and racing to Carnegie R1 status.

TXST is strategically located in the Texas Triangle, the dynamic regions
of Texas that encompasses San Antonio, Austin, Dallas-Forth Worth, and
Houston(\textbf{txcampplan24?}). This area is home to more than 70\% of
Texas population(\textgreater21 million people) and 80\% of Texas GDP.
The Texas Triangle has experienced rapid growth, reaching over 15\% in
the last decade. This setting provides TXST school system with direct
connection to a diverse pool of talent, proximity to leading industries,
and a vibrant cultural and economic environment. With seven distinct
locations along the I-35 corridor between Austin and San Antonio, TSXT
is uniquely positioned to serve the growing region.

San Marcos and Round Rock campuses benefit from proximity to two of
Texas's fastest-growing metropolitan areas providing a robust enrollment
catchment zone and direct access to diverse industries like technology,
healthcare, and manufacturing. These campus locations are ideal for
building partnerships and fostering research and innovation. Along with
the main campus(San Marcos) TXST has Science, Technology, and Advanced
Research(STAR Park), Advanced Law Enforcement Rapid Response
Training(ALERRT), Muller Farm, Freeman Ranch, and University Camp. These
sites support specific functions for TXST and collectively enhance the
institution's research resources. STAR Park, which serves as an
extension of the university as a research hub for enterprise and
collaboration with industry partners. Muller Farm and Freeman Ranch
serve as hands on leaning environments for agriculture and land
management research. University Camp about 14 miles from main campus
along the Blanco river offers a recreational retreat and outdoor leaning
opportunities. These locations are part of the larges TXST systems and
are designed to work to benefit TXST San Macros campus population.

San Marcos, Texas, is a vibrant nexus within the rapidly growing I-35
corridor. In 2023 the population was over 71,000, nearly doubling since
2000. TXST San Marcos 517 acre main campus sits in the heart of San
Marcos, 3 blocks from downtown. As one of the largest employers in the
area TXST plays a significant role in the region, as campus population
contribute labor and consumers for local economy. San Marcos is often
recognized as a noble college town, with TXST students accounting for
more than half of the local population. Campus is compact, operating at
around 75 assignable square foot(ASF) per full-time equivalent(FTE),
suggesting a highly efficient use of space(\textbf{txcampplan24?}). San
Marco's dense urban landscape and TXST's compact campus create a
distinct opportunity for employing and assessing novel transportation
interventions.

The San Marcos Campus is situated atop rolling hills that creates a
unique environment adjacent to the San Marcos River and Downtown San
Marcos. The campus's unique connection to its surrounding ecology and
urban context provide a memorable experiences. The campus features a
200-foot elevation change, descending from northwest to low-lying areas
near Spring Lake and the San Marcos River. Old Main, perched on
Chautauqua Hill overlooking the river gives a notice of the hilly
topography. The San Marcos River and Spring Lake as a part of campus are
crucial natural features that provide scenic views and educational
opportunities. The university connection to these nature features is
emphasized in tradations, such as joining your cohort at Swell park to
jumping in the river after graduation.

Local Transportation assets and infrastructure

Pedestrain Circulation: TXST's campus is a pedestrian-oriented are
defined by three major east-west pedestrian axes: The Quad, Bobcat
Trail, and Concho Green. TXST has a core

Parking: The university maintains a comprehensive parking network with
parking garages, surfaces lots, and on-street parking. Nine parking
garages are concentrated in the west and central areas near Chautauqua.
However, rapid growth has created a shortage in most parking categories.

Roadways and Gateways: Key intersections like Aquarena Springs Drive and
Sessom Drive serve as major gateways but experience significant
congestion during peak hours. Strategies to improve traffic flow are
recommended, including low-impact solutions like re striping and signal
timing, though physical modifications are limited by the river and
existing structures. Enhancements to streets capes and signage are
proposed at gateways to strengthen the universities presence.

Transit and Mobility: The Campus Plan prioritizes enhancing physical
connectivity through pedestrian pathways, transit, and micro-mobility.
Future-oriented solutions envision advanced transit systems like
enhanced bus services(express routes,dedicated lanes), autonomous
shuttles, personal rapid transit, streetcars, or light rail in
collaboration with the City of San Marcos. The plan also includes shared
bikes paths and tracks. There's also interest in making alternative
modes more inviting, accessible, and safe to reduce car dependency.

Connectivity Projects: A proposed pedestrian bridge connecting the
university core pedestrian corridor to the STEM quad area is discussed
to provide connection. This project would require City of San Marcos
review and permitting.

High Density Transportation: The high density at Texas State increases
the potential for transport infrastructure and services supporting
multi-modal travel. A compact campus design can promote high trip rates
among students. High student density present challenges but also
opportunities for promoting alternative transportation. Analyzing the
travel behavior using parking survey data is an ongoing research effort
to leverage campus characteristics and existing multi-modal systems.

In summary TXST and San Marcos offer a compelling location for research,
situated within a rapidly growing economic hub and benefiting from the
university's race to R1 and its network of specialized satellite
locations. The San Marcos campus itself is a characterized by its size,
unique topography, and integration with significant natural features
like the San Marcos River. The local transportation environment includes
a pedestrian-centric core, existing parking infrastructure, and
congested roadways challenges. The future vision includes enhancing
physical connectivity through transit, micro-mobility, and pedestrian
infrastructure improvements, addressing the challenges and capitalizing
on the opportunities presented by the compact, high density campus
environment. This research focuses on understanding how to further
leverage physical features of campus to reduce transportation emissions.

\subsection{Data Collection:}\label{data-collection}

The primary source for this research is Texas State University's Parking
survey from Fall 2024 and 2025-2035 Camps Master Plan.

The 2024 parking survey will be utilized to investigated TXST students
travel behavior. This parking survey was administrator via university
emails to all TXST students, gathering comprehensive information
regarding daily commuting patterns. This survey received a response rate
of 2.9\%(0.02908), The data collected from the student responses is a
quantitative in nature and will be used for a statistical summary
analysis of travel behavior.

\begin{itemize}
\tightlist
\item
  Analysis:

  \begin{itemize}
  \tightlist
  \item
    Perform statistical summary analysis of travel behavior based on
    survey responses.
  \item
    Identify key insights related to transportation patterns.

    \begin{itemize}
    \tightlist
    \item
      Primary travel mode
    \item
      Frequency of travel
    \item
      Trip distance and length
    \item
      Parking permit ownership
    \item
      Private vehicle access
    \end{itemize}
  \item
    Examine the potential impact of changes in travel behavior on
    greenhouse gas emissions.
  \end{itemize}
\item
  Summary and Recommendations:

  \begin{itemize}
  \tightlist
  \item
    Draft a summary analysis of the findings.
  \item
    Propose actionable steps for TXST to promote sustainable
    transportation.
  \end{itemize}
\end{itemize}

\subsubsection{Objectives}\label{objectives}

\begin{itemize}
\item
  Analyze transportation and travel behavior of Texas State University
  students using parking survey data from the Fall 2024 semester.
\item
  Identify trends and areas for improvement in campus transportation
  planning using 2025-2035 Campus Master Plans.
\item
  Provide actionable recommendations to reduce transportation-related
  greenhouse gas emissions.
\end{itemize}

\subsection{Expected Outcomes}\label{expected-outcomes}

\begin{itemize}
\item
  Enhanced understanding of TXST students' travel behavior and its
  environmental impact.
\item
  Identification of key factors and trends influencing transportation
  choices among students.
\item
  Actionable recommendations for TXST to reduce transportation-related
  GHGe and promote sustainable travel options.
\end{itemize}

\phantomsection\label{refs}
\begin{CSLReferences}{1}{0}
\bibitem[\citeproctext]{ref-deangelis2021}
Angelis, Marco De, Luca Mantecchini, and Luca Pietrantoni. 2021. {``A
Cluster Analysis of University Commuters: Attitudes, Personal Norms and
Constraints, and Travel Satisfaction.''} \emph{International Journal of
Environmental Research and Public Health} 18 (9): 4592.
\url{https://doi.org/10.3390/ijerph18094592}.

\bibitem[\citeproctext]{ref-khattak2011}
Khattak, Asad, Xin Wang, Sanghoon Son, and Paul Agnello. 2011. {``Travel
by University Students in Virginia: Is This Travel Different from Travel
by the General Population?''} \emph{Transportation Research Record:
Journal of the Transportation Research Board} 2255 (1): 137--45.
\url{https://doi.org/10.3141/2255-15}.

\bibitem[\citeproctext]{ref-parsley}
Parsley, Emma Corinne, and Tina Marie Waliczek. n.d. {``Measuring the
Perceptions of Sustainability Initiatives Within a University Campus.''}

\bibitem[\citeproctext]{ref-romanowska2019}
Romanowska, Aleksandra, Romanika Okraszewska, and Kazimierz Jamroz.
2019. {``A Study of Transport Behaviour of Academic Communities.''}
\emph{Sustainability} 11 (13): 3519.
\url{https://doi.org/10.3390/su11133519}.

\bibitem[\citeproctext]{ref-taylor2021}
Taylor, Ryan, and Raktim Mitra. 2021. {``Commute Satisfaction and Its
Relationship to Post-Secondary Students' Campus Participation and
Success.''} \emph{Transportation Research Part D: Transport and
Environment} 96: 102890.
\url{https://doi.org/10.1016/j.trd.2021.102890}.

\bibitem[\citeproctext]{ref-zhou2012}
Zhou, Jiangping. 2012. {``Sustainable Commute in a Car-Dominant City:
Factors Affecting Alternative Mode Choices Among University Students.''}
\emph{Transportation Research Part A: Policy and Practice} 46 (7):
1013--29. \url{https://doi.org/10.1016/j.tra.2012.04.001}.

\bibitem[\citeproctext]{ref-zhou2018}
Zhou, Jiangping, Yin Wang, and Jiangyue Wu. 2018. {``Mode Choice of
Commuter Students in a College Town: An Exploratory Study from the
United States.''} \emph{Sustainability} 10 (9): 3316.
\url{https://doi.org/10.3390/su10093316}.

\end{CSLReferences}




\end{document}
